%----------------------------------------------------------------------------------------
%	PACKAGES AND OTHER DOCUMENT CONFIGURATIONS
%----------------------------------------------------------------------------------------

\documentclass[paper=a4, fontsize=11pt]{scrartcl} % A4 paper and 11pt font size
\usepackage[a4paper, left=2.5cm, right=2cm, top=2cm, bottom=2cm]{geometry}
\linespread{1.2}

\usepackage[utf8]{inputenc}
\addtokomafont{disposition}{\rmfamily}
\usepackage{polski} % Język polski/hyphenation
\usepackage{amsmath,amsfonts,amsthm} % Math packages
\usepackage{physics}
\usepackage{marvosym}
\usepackage{wasysym}
\usepackage{indentfirst}
\usepackage{graphicx}
\usepackage{caption}




\def \nazwa {Praca licencjacka - Aplikacja uczenia maszynowego metodą SVM\\ \textbf{BRUDNOPIS}}
\def \autor {Pavlo Boidachenko}




\usepackage{lipsum} % Used for inserting dummy 'Lorem ipsum' text into the template

\usepackage{sectsty} % Allows customizing section commands
%\allsections{\mdseries\itshape} % Make all sections centered, the default font and small caps

\usepackage{fancyhdr} % Custom headers and footers
\pagestyle{fancy} % Makes all pages in the document conform to the custom headers and footers
\fancyhead[L]{\textbf{Praca licencjacka:} Aplikacja uczenia maszynowego metodą SVM} % No page header - if you want one, create it in the same way as the footers below
\fancyhead[C]{}
\fancyhead[R]{}
\fancyfoot[L]{\autor} % left footer
\fancyfoot[C]{\thepage} % center footer
\fancyfoot[R]{} % right footer
\renewcommand{\headrulewidth}{0.3pt} % Remove header underlines
\renewcommand{\footrulewidth}{0.3pt} % Remove footer underlines
\setlength{\headheight}{14.06pt} % Customize the height of the header

\usepackage[style=alphabetic,sorting=nyt,sortcites=true,autopunct=true,babel=hyphen,hyperref=true,abbreviate=false,backref=true,backend=biber]{biblatex}


\numberwithin{equation}{section} % Number equations within sections (i.e. 1.1, 1.2, 2.1, 2.2 instead of 1, 2, 3, 4)
\numberwithin{figure}{section} % Number figures within sections (i.e. 1.1, 1.2, 2.1, 2.2 instead of 1, 2, 3, 4)
\renewcommand{\thetable}{\Roman{table}}

\setlength\parindent{10pt} % Removes all indentation from paragraphs - comment this line for an assignment with lots of text

\newcommand{\horrule}[1]{\rule{\linewidth}{#1}} % Create horizontal rule command with 1 argument of height

%Other packages
\usepackage{multirow}
\usepackage{url}
\usepackage{bm}
\usepackage{xcolor}
\definecolor{Black}{RGB}{0,0,0}
\definecolor{Red}{RGB}{255,0,0}
\definecolor{Blue}{RGB}{0,0,255}
\definecolor{Green}{RGB}{0,255,0}
\definecolor{Gray}{RGB}{45,45,45}
\definecolor{linkcol}{RGB}{57,0,155}
\usepackage[unicode, pdftex, colorlinks=true, urlcolor=Gray, linkcolor=Gray, citecolor=Gray]{hyperref}

\usepackage{tocloft}
\renewcommand{\cftsecleader}{\cftdotfill{\cftdotsep}}

\title{	
\normalfont \normalsize 
Uniwersytet Jagielloński\\Wydział Fizyki, Astronomii i Informatyki Stosowanej\\Instytut - Obserwatorium Astronomiczne \\ [20pt] % Your university, school and/or department name(s)
\horrule{0.5pt} \\[0.4cm] % Thin top horizontal rule
\Large Pracownia astronomii praktycznej \\  \nazwa \\
\horrule{2pt} \\[0.5cm] % Thick bottom horizontal rule
}

\author{\autor\\} % Your name

\date{\data} % Today's date or a custom date

\begin{document}
\thispagestyle{empty}
\begin{titlepage}
    \begin{center}

           \Large
	\textbf{Uniwersytet Jagielloński w Krakowie}\vspace{0.2cm}\\ Wydział Fizyki, Astronomii i Informatyki Stosowanej
               \vspace*{1cm}
               
         \vspace{3cm}
         \Large
          \textbf{Pavlo Boidachenko}\\\vspace{0.5cm}
         \normalsize Nr albumu: 1124969\\
             \vspace{2cm}
        \Huge
        \textbf{Aplikacja uczenia maszynowego metodą SVM}
      
        \vspace{1.5cm}
        \normalsize
        Praca licencjacka\\
        na kierunku informatyki\\ \vspace{0.15cm}
        
        \vfill
        \vspace{2cm}
       \begin{minipage}{1\textwidth}
\begin{flushright}
Praca wykonana pod kierunkiem\\
dr Grzegorz Surówka\\
\end{flushright}
\end{minipage}
        
        \vspace{2cm}
        \begin{center}
      Kraków 2019
        \end{center}
    \end{center}
\end{titlepage}

\newpage 
 \thispagestyle{empty}
\vspace{2.5cm}
\begin{flushleft}
\large \textbf{Oświadczenie autora pracy}\vspace{0.6cm}\\
\end{flushleft}

\noindent Świadom odpowiedzialności prawnej oświadczam, że niniejsza praca dyplomowa została napisana przeze mnie samodzielnie i nie zawiera treści uzyskanych w sposób niezgodny z obowiązującymi przepisami.\\

\noindent Oświadczam również, że przedstawiona praca nie była wcześniej przedmiotem procedur związanych z uzyskaniem tytułu zawodowego w wyższej uczelni.
\vspace{2cm}
\begin{center}
    \begin{tabular}{lr}
        ................................~~~~~~~~~~~~~~~~~~~~~~~~~~~~~~~~~~~~~~&
        .......................................... \\
        {~~~~Kraków, dnia} & {Podpis autora pracy~~~~}
    \end{tabular}
\end{center}
\vspace{5cm}
\begin{flushleft}
    \large \textbf{Oświadczenie kierującego pracą}
\end{flushleft}

\noindent Potwierdzam, że niniejsza praca została przygotowana pod moim kierunkiem i~kwalifikuje się do przedstawienia jej w postępowaniu o nadanie tytułu zawodowego.
\vspace{2cm}
\begin{center}
    \begin{tabular}{lr}
        ................................~~~~~~~~~~~~~~~~~~~~~~~~~~~~~~~~~~~~~~&
        ............................................ \\
        {~~~~Kraków, dnia} & {Podpis kierującego pracą~~}
    \end{tabular}
\end{center}
\vfill
\newpage
\tableofcontents
\newpage
\section{Wstęp}
\subsection{Motywacja}
\subsection{Cel}
\subsection{Zakres}
\section{Metoda klasyfikacji SVM}
\subsection{Opis}
\subsection{C-SVM}
\subsection{$\mu$-SVM}
\subsection{Kernel}
\section{Projekt aplikacji}
\subsection{Opis}
\subsection{Technologie}
\subsection{}
\section{Podsumowanie}
\subsection{Odniesienie do celu pracy}
\subsection{Co można dodać}

\end{document}
